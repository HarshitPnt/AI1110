\documentclass{beamer}
\usetheme{CambridgeUS}
\usepackage{enumerate}
\usepackage{amsmath,amsfonts,amssymb}
\usepackage{setspace}
%\usepackage{gensymb}
\usepackage{subfig}
\usepackage{float}
\usepackage{cellspace}
\usepackage{longtable}
\usepackage{hhline}
\def\inputGnumericTable{}
\usepackage{bigstrut}
\newcounter{saveenumerate}
\newcommand{\saveenumerate}{\setcounter{saveenumerate}{\value{enumi}}}
\newcommand{\restartenumerate}{\setcounter{enumi}{\value{saveenumerate}}}
\providecommand{\brak}[1]{\ensuremath{\left(#1\right)}}
\setlength{\abovedisplayskip}{0in}
\title{Assignment 8} 
\author[CS21BTECH11021]{Harshit Pant\\CS21BTECH11021}
\date{}
\providecommand{\pr}[1]{\ensuremath{\Pr\left(#1\right)}}
%\logo{\large \LaTeX{}}
\begin{document}
\begin{block}
\titlepage
\end{block}
\begin{frame}{Question}
\begin{block}{Question:}
The resistors $r_1$,$r_2$,$r_3$ and $r_4$ are independent random variables and each is uniform in the interval $(450,550)$. Using the central limit theorem, find $\pr{1900\leq r_1+r_2+r_3+r_4\leq 2100}$.
\end{block}
\end{frame}
\begin{frame}{CLT}
\begin{block}{Central limit Theorem:}
If $X_1,X_2,\ldots,X_n\ldots$ is a sequence of random variables drawn from a population with an overall mean $\mu$ and variance $\sigma^2$, and if $\overline{X_n}$ is the sample mean of the first $n$ samples, then the limiting form of the distribution, $Z=\displaystyle{\lim\limits_{n\to\infty}\brak{\dfrac{\overline{X_n}-\mu}{\frac{\sigma}{\sqrt{n}}}}}$, is a standard normal distribution.
\end{block}
\end{frame}
\begin{frame}{Solution:}
\begingroup
\addtolength{\jot}{.1in}
Since $r_i \,\forall i\in\{1,2,3,4\}$ are i.i.d and are uniformly distributed in the interval $(450,550)$,
\begin{align}
E(r_i)&=500\\
\sigma_i^2&=\int \limits_{-\infty}^\infty {(r-E(r_i))}^2\rho(r)dr\\
&=\int\limits_{450}^{550} {(r-E(r_i))}^2\times \frac1{100}dr\\
&=\int\limits_{450}^{550} \frac{{(r-500)}^2}{100}dr\\
\end{align}
\endgroup
\end{frame}
\begin{frame}{Solution:}
\begingroup
\addtolength{\jot}{.1in}
\begin{align}
&=\int\limits_{-50}^{50} \frac{r^2}{100}dr\\
&=\frac{50^2}{3}
\end{align}
Let,
\begin{align}
Y&=\sum_1^4 r_i
\end{align}
Using Central Limit Theorem:
\begin{align}
f_Y(y)&\sim N\brak{4\times500,\frac{2\times50}{\sqrt{3}}}
\end{align}
\endgroup
\end{frame}
\begin{frame}
\begin{figure}[H]
\centering
\includegraphics[scale=0.5]{./PDF_assign8.png}
\caption{PDF of R}
\end{figure}
\end{frame}
\begin{frame}
\begingroup
\addtolength{\jot}{.1in}
\begin{align}
\pr{1900\leq y \leq 2100}&=\int\limits_{1900}^{2100}f_Y(y)dy\\
&=\int\limits_{1900}^{2100} \dfrac{1}{\sigma\sqrt{2 \pi}}e^{-{(\frac{y-\mu}{\sqrt{2}\sigma})}^2}dy\\
&=\int\limits_{-100}^{100} \dfrac{1}{\sigma\sqrt{2 \pi}}e^{-\frac{y^2}{2\sigma^2}}dy\\
&\approx0.91673
\end{align}
\endgroup
\end{frame}
\end{document}