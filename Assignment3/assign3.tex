\documentclass[journal,12pt,twocolumn]{IEEEtran}
\usepackage{amsmath,amsfonts,amssymb}
\usepackage{setspace}
\usepackage{float}
\usepackage{cellspace}
\usepackage{longtable}
\usepackage{bigstrut}
\usepackage{ifthen}
\providecommand{\brak}[1]{\ensuremath{\left(#1\right)}}
\title{Assignment 3}
\author{Harshit Pant\\CS21BTECH11021}
\date{}
\def\inputGnumericTable{}
\begin{document}
\maketitle
\begin{abstract}
This document contains the solution to Example 12 of Chapter 14 (Statistics) in the NCERT Class 9 Exemplar.
\end{abstract}
\textbf{Example 12:}	The heights (in cm) of 9 students of a class are as follows:\\

\begin{table}[!htb]
\input{raw.tex}
\caption{Raw Data}
\label{table:raw}
\end{table}
Find the median height of this data.\\

\textbf{Solution: }In order to calculate the median of this data, we need to arrange it in ascending order,\\
\begin{table}[!htb]
\centering
\input{sort.tex}
\caption{Sorted Data}
\label{table:sort}
\end{table}

Since the number of students is an odd number, the median height is equal to the height of $\brak{\frac{9+1}{2}}^{\text{th}}$ student in this sorted table.
Hence the median height of the students is $149\,$cm.
\end{document}