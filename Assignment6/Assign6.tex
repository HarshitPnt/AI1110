\documentclass{beamer}
\usetheme{CambridgeUS}
\usepackage{enumerate}
\usepackage{amsmath,amsfonts,amssymb}
\usepackage{setspace}
\usepackage{float}
\usepackage{cellspace}
\usepackage{longtable}
\usepackage{hhline}
\def\inputGnumericTable{}
\usepackage{bigstrut}
\newcounter{saveenumerate}
\newcommand{\saveenumerate}{\setcounter{saveenumerate}{\value{enumi}}}
\newcommand{\restartenumerate}{\setcounter{enumi}{\value{saveenumerate}}}
\setlength{\abovedisplayskip}{0in}
\title{Assignment 6} 
\author[CS21BTECH11021]{Harshit Pant\\CS21BTECH11021}
\date{}
\providecommand{\pr}[1]{\ensuremath{\Pr\left(#1\right)}}
%\logo{\large \LaTeX{}}
\begin{document}
\begin{frame}
    \titlepage 
\end{frame}
\begin{frame}{Question}
    %\tableofcontents
%\begin{abstract}
%This document contains the solution to Example 2.15  of Chapter 2 (The Axioms Of Probabiltiy) from the book Probability, Random Variables, and Stochastic Processes by Athanasios Papoulis.
%\end{abstract}
\begin{block}{\textbf{Example 2.15 [Papoulis]:}}
 A certain test for a particular cancer is known to be 95$\%$ accurate. A person submits to the test and the results are positive. Suppose that the person comes from a population of 100,000 where 2,000 people suffer from that disease. What can we conclude about the probability that the person under test has that particular cancer?
 \end{block}
\end{frame}
\begin{frame}{Answer}
\begin{exampleblock}{}
Let $X_1$ and $X_2$ be two Bernoulli Random Variables such that
\end{exampleblock}
\begin{table}[H]
\centering
\input{table1.tex}
\caption{Bernoulli Distribution}
\label{table:RV1}
\end{table}
\begin{table}[H]
\centering
\input{table2.tex}
\caption{Bernoulli Distribution}
\label{table:RV2}
\end{table}
\end{frame}
\begin{frame}{Answer}
\begin{block}{Given:}
\begin{align}
\pr{X_1=1}&=0.02\\
\pr{X_2=1|X_1=1}&=0.95\\
\pr{X_2=0|X_1=1}&=0.05\\
\pr{X_2=0|X_1=0}&=0.95
\end{align}
\end{block}
\begingroup
\addtolength{\jot}{.1in}
\begin{align}
\pr{X_1=1|X_2=1}&=\dfrac{\pr{X_2=1|X_1=1}\pr{X_1=1}}{\pr{X_2=1}}\\
\pr{X_2=1}&=\sum_{i=0}^1 \pr{X_2=1|X_1=i}\pr{X_1=i}
\end{align}
\endgroup
\end{frame}
\begin{frame}{Answer}
\begingroup
\addtolength{\jot}{.1in}
\begin{align}
\pr{X_2=1}&=0.05\times0.98 +0.95\times0.02\\
\pr{X_2=1}&=0.068\\
\pr{X_1=1|X_2=1}&=\dfrac{0.95\times0.02}{0.068}\\
\pr{X_1=1|X_2=1}&=\dfrac{0.019}{0.068}\\
\pr{X_1=1|X_2=1}&=0.2794
\end{align}
\endgroup
Baye's theorem suggests that there is a $\fbox{27.94\%}$ chance that this person has cancer given he/she has been tested positive.
\end{frame}
\end{document}